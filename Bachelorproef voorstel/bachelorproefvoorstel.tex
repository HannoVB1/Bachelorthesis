%==============================================================================
% Voorbeeld hogent-article: onderzoeksvoorstel bachproef
%==============================================================================

\documentclass{hogent-article}

% Invoegen bibliografiebestand
\addbibresource{bronnen.bib}

% Informatie over de opleiding, het vak en soort opdracht
\studyprogramme{Professionele bachelor toegepaste informatica}
\course{Bachelorproef}
\assignmenttype{Onderzoeksvoorstel}
\academicyear{2023-2024}

\title{Heeft de maancyclus een invloed op het gedrag van de cryptocurrency markt?: een onderzoek naar een correlatie tussen de maancyclus en de cryptomarkt door gebruik te maken van AI.}

\author{Hanno van Baarle}
\email{hanno.vanbaarle@student.hogent.be}

% Gaat het om een bachelorproef in samenwerking met een student in een andere
% opleiding? Geef dan de naam en emailadres hier
% \author{/}
% \email{/}

% TODO: Geef de co-promotor op
%\supervisor[Co-promotor]{S. Beekman (Synalco, \href{mailto:sigrid.beekman@synalco.be}{sigrid.beekman@synalco.be})}

% Binnen welke specialisatierichting uit 3TI situeert dit onderzoek zich?
% Kies uit deze lijst:
%
% - Mobile \& Enterprise development
% - AI \& Data Engineering
% - Functional \& Business Analysis
% - System \& Network Administrator
% - Mainframe Expert
% - Als het onderzoek niet past binnen een van deze domeinen specifieer je deze
%   zelf
%
\specialisation{AI \& Data Science}
\keywords{AI, Cryptocurrency, Blockchain, Mooncycles, Financiën}

\begin{document}

\begin{abstract}
\end{abstract}

\tableofcontents

\section{Inleiding}%
\label{sec:inleiding}

In de snel evoluerende wereld van cryptocurrency's wordt gezocht naar innovatieve benaderingen om het gedrag van de markt beter te begrijpen en voorspellingen te verbeteren. Een ongewoon doch intrigerend aspect dat de aandacht heeft getrokken, is de maancyclus. Dit onderzoek tracht te onderzoeken of er al dan niet een aantoonbare correlatie bestaat tussen de verschillende fasen van de maancyclus en het gedrag van de cryptomarkt.

De maancyclus, die zich over een periode van ongeveer 29,5 dagen voltrekt, heeft eeuwenlang fascinatie gewekt en is vaak in verband gebracht met diverse aspecten van het menselijk leven en de natuur. Zo was de maan bijvoorbeeld een teken van tijd, vruchtbaarheid en groei in het oude egypte \cite{MoonCraterTycho2023}. Dit onderzoek gaat een stap verder door gebruik te maken van geavanceerde artificiële intelligentie (AI) om patronen te identificeren en mogelijke correlaties tussen de maancyclus en de cryptomarkt te analyseren.

Door middel van geautomatiseerde data-analyse, machine learning-algoritmen en diepgaande marktgegevens, beoogt dit onderzoek niet alleen de aanwezigheid van correlaties te identificeren, maar ook inzichten te krijgen in hoe dergelijke cyclische patronen invloed zouden kunnen hebben op het gedrag van de cryptomarkt. Dit zou niet alleen van academisch belang kunnen zijn, maar ook praktische implicaties kunnen hebben voor beleggers en marktdeelnemers die trachten de volatiliteit van cryptocurrency's beter te begrijpen en te benutten.

Voorgaande studies hebben reeds onderzoek gedaan naar de potentiële invloed van de maan op zowel de aandelen- als de cryptomarkt. Echter, deze studies concentreerden zich specifiek op de prijsbewegingen van individuele aandelen of cryptocurrencies en verkenden niet de bredere marktinvloeden. Het onderscheidende kenmerk van deze studie ligt in haar streven om verder te gaan dan enkel prijsanalyses. In plaats daarvan richt deze studie zich op een uitgebreidere reeks parameters, waaronder handelsvolume (TV), de 'greed-index', en Bitcoin-dominantie. Door deze brede benadering beoogt de studie niet alleen de invloed van de maan op prijzen te onderzoeken, maar ook op de algehele marktdynamiek. Dit streven naar een meer omvattende analyse positioneert deze studie als een unieke bijdrage aan het bestaande onderzoek op het raakvlak van maancycli en financiële markten.

\section{Literatuurstudie}%
\label{sec:Literatuurstudie}

ookal heeft de studie een focus op de volledige markt blijft het waardevol om eerdere studies met betrekking tot specifieke cryptocurrencies, zoals Bitcoin en andere `digital currencies`, grondig te onderzoeken. 

In 2023 deden 4 studenten aan de universiteit van Hacettepe, Ugurcan Erdogan, Alperen Berk Isildar, Tugba Gurgen Erdogan en Fuat Akal \cite{Marquez2023} een studie rond de prijs impact van lunar cycles op bitcoin. Ze maakten hier gebruik van McNemar's Chi-Square test om onderzoek te doen. In hun studie concludeerden de studenten dat er volgens de Chi-Square test hoogstwaarschijnlijk geen duidelijk correlatie is tussen de prijs van Bitcoin en de cycli van de maan.

Een andere studie in 2014 vond een tegenstrijdige conclusie waaruit bleek dat beleggerspsychologie wordt beïnvloed door de volle maan, maar er werd geen effect waargenomen tijdens de nieuwe maanfase. Bevestigd door de paired t-difference test, de kleine correlatie, naast het kwantitatieve model, tonen de resultaten aan dat de volle maan invloed heeft op het marktgedrag tijdens haar `orbital phase`. Als gevolg hiervan veronderstellen de auteurs dat de volle maan daadwerkelijk van invloed is op de cognitie van beleggers en emotionele verstoring, stemmingsstoornissen en agressiviteit, wat resulteert in een slechte prestatie bij het handelen in aandelen.
\textcite{Brahmana2014}

\section{Methodologie}%
\label{sec:Methodologie}

Hoewel veel sceptici zeiden dat het niet mogelijk was (met name Wilson en Moore), beschrijven wij een volledig werkende versie van ons systeem. Bovendien, was het nodig om de instructie snelheid die door ons raamwerk wordt gebruikt te beperken tot 30 pagina's. De met de hand geoptimaliseerde compiler bevat ongeveer 403 instructies van ML. Onze oplossing bestaat uit een client-side bibliotheek, een verzameling shell scripts, en een eigen database. Evenzo bevat de verzameling van shell scripts bevat ongeveer 52 instructies van~\textcite{SabiEtAl2016}. Over het geheel genomen, voegt YnowHip slechts bescheiden overhead en complexiteit toe aan eerdere adaptieve systemen.

\section{Verwacht resultaat, conclusie}%
\label{sec:Resultaat-en-conclusie}


\printbibliography[heading=bibintoc]

\end{document}